
\newcounter{publicationCounter}			% We set an publication counter, we can continue counting the publications when switchen from conferences to workshop etc.
\subsection{Job Market Paper}

\textbf{Structural adjustments and food sovereignty: The effects of IMF
programs and loan conditions on food self-sufficiency.}
\vspace{1em}

 \parbox{\textwidth}{In this paper I study the effects of IMF program participation and associated policy conditions, known as ‘conditionalities,’ on food self-sufficiency of IMF members for the period of 1980-2019. In the empirical analysis, I employ a compound instrumental variable approach within a system of three equations, estimated using maximum likelihood. After accounting for confounding economic and climatic factors and addressing selection bias, I find no robust evidence that IMF program participation has any effect on food self-sufficiency. In contrast, program conditions have a significant effect on food self sufficiency. Based on point estimates, I find that each additional binding IMF condition reduces the food self-sufficiency ratio by 0.003 percentage points. Disaggregating conditions by policy area, the findings suggest that IMF conditionality hinders food self-sufficiency, with policy reforms in trade and exchange systems, financial sectors, monetary policy, and central banking playing a particularly significant role.}

\subsection{Working Papers}

\textbf{Revisiting the pollution haven hypothesis and trade-embodied emissions: Accounting for global value chains.}
\vspace{1em}

 \parbox{\textwidth}{This paper investigates the empirical link between environmental policy and trade flows by incorporating global value chain in the analysis. To do so, I use $CO_{2}$ embodied in domestic final demand as an indicator of the pollution content of imports (PCI), which is corrected for trade in value-added flows. I begin with investigation of North-South patterns, in particular if the PCI is relatively larger in North countries. Then, I carry out analysis of the determinants of PCI using a gravity-based empirical framework to quantitatively measure the impact of Pollution Haven (PH) and Factor endowment (FE) effects. Using several proxies for policy stringency, I show that PCI is higher for the North countries, which have more stringent regulations.  PH effect is found to be larger than what previous studies find, while FE has the opposite effect on PCI of the North. The former dominates the latter, together confirming the pollution haven hypothesis.}

\vspace{1em}
\textbf{How international sanctions shape the carbon footprint of global trade.}
\vspace{1em}

 \parbox{\textwidth}{This paper investigates the overlooked environmental consequences of international sanctions by empirically examining their impact on trade-embodied carbon emissions. While sanctions are traditionally assessed through their economic or political effects, this study highlights how such measures can inadvertently shift pollution- intensive production and trade flows toward regions with laxer environmental regulations, thereby increasing global emissions. Using comprehensive sanction records for 212 countries from 1950-2022, the analysis employs a gravity-based model, estimated using maximum likelihood, to examine the effect of sanctions on emissions embodied in trade. By integrating sanctions into the broader climate and trade policy discourse, the paper offers novel insights into how foreign policy tools may unintentionally undermine global environmental goals.}




